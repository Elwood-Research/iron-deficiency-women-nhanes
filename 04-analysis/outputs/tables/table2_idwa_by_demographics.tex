\begin{table}[htbp]
\centering
\caption{Prevalence of Iron Deficiency Without Anemia by Demographic Characteristics}
\label{tab:table2}
\begin{tabular}{lccc}
\toprule
\textbf{Characteristic} & \textbf{n/N} & \textbf{Prevalence, \\% (SE)} \\
\midrule
\textbf{Overall} & & \\
\quad All & 580/6125 & 9.0 (0.4) \\
\midrule
\textbf{Age Group} & & \\
\quad 18-25 & 209/1859 & 10.0 (0.7) \\
\quad 26-30 & 72/1021 & 7.3 (0.8) \\
\quad 31-35 & 76/988 & 7.1 (0.8) \\
\quad 36-40 & 109/1088 & 10.3 (0.9) \\
\quad 41-45 & 114/1169 & 9.5 (0.9) \\
\midrule
\textbf{Race/Ethnicity} & & \\
\quad Mexican American & 157/1264 & 11.6 (0.9) \\
\quad Other Hispanic & 68/663 & 10.1 (1.2) \\
\quad Non-Hispanic White & 204/2227 & 8.8 (0.6) \\
\quad Non-Hispanic Black & 91/1317 & 6.5 (0.7) \\
\quad Other Race & 60/654 & 10.0 (1.2) \\
\midrule
\textbf{Poverty Status} & & \\
\quad Low (<1.3) & 224/2130 & 10.2 (0.7) \\
\quad Medium (1.3-3.5) & 189/2071 & 9.3 (0.6) \\
\midrule
\textbf{Iron Supplement} & & \\
\quad No & 500/5107 & 9.2 (0.4) \\
\quad Yes & 80/1018 & 7.9 (0.8) \\
\bottomrule
\end{tabular}
\begin{flushleft}
\footnotesize{\textit{Note:} n = number with IDWA; N = total in subgroup. SE = standard error.}
\end{flushleft}
\end{table}