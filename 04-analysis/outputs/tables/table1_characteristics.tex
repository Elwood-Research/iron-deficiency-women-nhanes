\begin{table}[htbp]
\centering
\caption{Characteristics of Study Population}
\label{tab:table1}
\begin{tabular}{lc}
\toprule
\textbf{Characteristic} & \textbf{Value} \\
\midrule
Sample size, n & 6,125 \\
\midrule
\textbf{Age, years} & \\
\quad Mean (SD) & 32.1 (8.1) \\
\quad Median [IQR] & 31.0 [24.0, 39.0] \\
\midrule
\textbf{Race/Ethnicity, \\% (SE)} & \\
\quad Mexican American & 10.9 (0.4) \\
\quad Other Hispanic & 7.1 (0.3) \\
\quad Non-Hispanic White & 60.0 (0.6) \\
\quad Non-Hispanic Black & 13.0 (0.4) \\
\quad Other Race & 9.0 (0.4) \\
\midrule
\textbf{Poverty Status} & \\
\quad Poverty ratio, mean (SD) & 2.7 (1.6) \\
\quad Poverty category, \\% (SE) & \\
\quad \quad Low (<1.3) & 25.2 (0.6) \\
\quad \quad Medium (1.3-3.5) & 34.3 (0.6) \\
\quad \quad High (>=3.5) & 0.0 (0.0) \\
\midrule
\textbf{BMI, kg/m\textsuperscript{2}, mean (SD)} & 28.4 (7.8) \\
\midrule
\textbf{Iron Status} & \\
\quad Ferritin, ng/mL, median [IQR] & 37.2 [20.0, 67.0] \\
\quad Hemoglobin, g/dL, mean (SD) & 13.3 (1.2) \\
\midrule
\textbf{IDWA prevalence, \% (SE)} & 9.0 (0.4) \\
Iron deficiency prevalence, \% (SE) & 15.1 (0.5) \\
Anemia prevalence, \% (SE) & 9.7 (0.4) \\
\midrule
\textbf{Iron supplement use, \% (SE)} & 18.5 (0.5) \\
\quad Iron dose category, \\% (SE) & \\
\quad \quad None & 0.0 (0.0) \\
\quad \quad Low & 9.4 (0.4) \\
\quad \quad Moderate & 5.5 (0.3) \\
\quad \quad High & 3.5 (0.2) \\
\bottomrule
\end{tabular}
\begin{flushleft}
\footnotesize{\textit{Note:} IDWA = Iron Deficiency Without Anemia. Values are weighted estimates unless otherwise noted. SE = standard error. IQR = interquartile range.}
\end{flushleft}
\end{table}