% ============================================
% TABLES FOR NHANES IRON DEFICIENCY STUDY
% ============================================
\documentclass[11pt]{article}
\usepackage[utf8]{inputenc}
\usepackage{booktabs}
\usepackage{array}
\usepackage{longtable}
\usepackage{caption}
\usepackage{geometry}

\geometry{margin=2.5cm}

\begin{document}

% ============================================
% TABLE 1: STUDY POPULATION CHARACTERISTICS
% ============================================
\begin{table}[htbp]
\centering
\caption{Characteristics of the Study Population (n=6,125)}
\label{tab:table1}
\begin{tabular}{lc}
\toprule
\textbf{Characteristic} & \textbf{Value} \\
\midrule
\multicolumn{2}{l}{\textbf{Demographics}} \\
\quad Age, years, mean (SD) & 32.1 (8.1) \\
\quad \quad 18--25 years, \% (SE) & 30.4 (0.6) \\
\quad \quad 26--30 years, \% (SE) & 16.7 (0.5) \\
\quad \quad 31--35 years, \% (SE) & 16.1 (0.5) \\
\quad \quad 36--40 years, \% (SE) & 17.8 (0.5) \\
\quad \quad 41--45 years, \% (SE) & 19.1 (0.5) \\
\midrule
\quad Race/Ethnicity, \% (SE) &  \\
\quad \quad Mexican American & 10.9 (0.4) \\
\quad \quad Other Hispanic & 7.1 (0.3) \\
\quad \quad Non-Hispanic White & 60.0 (0.6) \\
\quad \quad Non-Hispanic Black & 13.0 (0.4) \\
\quad \quad Other/Multiracial & 9.0 (0.4) \\
\midrule
\multicolumn{2}{l}{\textbf{Socioeconomic}} \\
\quad Poverty income ratio, mean (SE) & 2.7 (0.03) \\
\quad Low income ($<$1.3 PIR), \% (SE) & 25.2 (0.6) \\
\quad Medium income (1.3--3.5 PIR), \% (SE) & 34.3 (0.6) \\
\quad High income ($\geq$3.5 PIR), \% (SE) & 40.5 (0.6) \\
\midrule
\multicolumn{2}{l}{\textbf{Health Characteristics}} \\
\quad BMI, kg/m$^2$, mean (SD) & 28.4 (7.8) \\
\quad \quad Underweight ($<$18.5), \% (SE) & 2.1 (0.2) \\
\quad \quad Normal (18.5--24.9), \% (SE) & 33.8 (0.6) \\
\quad \quad Overweight (25.0--29.9), \% (SE) & 26.4 (0.6) \\
\quad \quad Obese ($\geq$30.0), \% (SE) & 37.7 (0.6) \\
\midrule
\multicolumn{2}{l}{\textbf{Iron Status}} \\
\quad Ferritin, ng/mL, median [IQR] & 37.2 [20.0, 67.0] \\
\quad \quad Geometric mean (95\% CI) & 38.6 [37.0, 40.3] \\
\quad Hemoglobin, g/dL, mean (SD) & 13.3 (1.2) \\
\quad Transferrin saturation, \%, mean (SD) & 24.8 (10.2) \\
\midrule
\multicolumn{2}{l}{\textbf{Prevalence Estimates}} \\
\quad Iron deficiency (ferritin $<$15), \% (95\% CI) & 15.1 [14.3, 15.9] \\
\quad Iron deficiency without anemia, \% (95\% CI) & 9.0 [8.3, 9.7] \\
\quad Anemia (Hb $<$12 g/dL), \% (95\% CI) & 9.7 [9.0, 10.4] \\
\midrule
\multicolumn{2}{l}{\textbf{Supplement Use}} \\
\quad Any iron supplement, \% (SE) & 18.5 (0.5) \\
\quad \quad Low dose ($<$18 mg/day), \% (SE) & 9.4 (0.4) \\
\quad \quad Moderate dose (18--27 mg/day), \% (SE) & 5.5 (0.3) \\
\quad \quad High dose ($\geq$28 mg/day), \% (SE) & 3.5 (0.2) \\
\bottomrule
\end{tabular}
\begin{flushleft}
\footnotesize{\textit{Notes:} Values are weighted estimates unless otherwise noted. SE = standard error; IQR = interquartile range; PIR = poverty income ratio; CI = confidence interval. IDWA = iron deficiency without anemia defined as ferritin $<$15~$\mu$g/L with hemoglobin $\geq$12~g/dL.}
\end{flushleft}
\end{table}

\newpage

% ============================================
% TABLE 2: IDWA PREVALENCE BY DEMOGRAPHICS
% ============================================
\begin{table}[htbp]
\centering
\caption{Prevalence of Iron Deficiency Without Anemia by Demographic Characteristics}
\label{tab:table2}
\begin{tabular}{lccc}
\toprule
\textbf{Characteristic} & \textbf{Sample Size} & \textbf{IDWA Cases} & \textbf{Prevalence, \% (95\% CI)} \\
\midrule
\textbf{Overall} & 6,125 & 580 & 9.0 [8.3, 9.7] \\
\midrule
\textbf{Age Group} & & & \\
\quad 18--25 years & 1,859 & 209 & 10.0 [8.7, 11.3] \\
\quad 26--30 years & 1,021 & 72 & 7.3 [5.7, 8.9] \\
\quad 31--35 years & 988 & 76 & 7.1 [5.5, 8.7] \\
\quad 36--40 years & 1,088 & 109 & 10.3 [8.5, 12.1] \\
\quad 41--45 years & 1,169 & 114 & 9.5 [7.8, 11.2] \\
\midrule
\textbf{Race/Ethnicity} & & & \\
\quad Mexican American & 1,264 & 157 & 11.6 [9.8, 13.4] \\
\quad Other Hispanic & 663 & 68 & 10.1 [7.7, 12.5] \\
\quad Non-Hispanic White & 2,227 & 204 & 8.8 [7.6, 10.0] \\
\quad Non-Hispanic Black & 1,317 & 91 & 6.5 [5.1, 7.9] \\
\quad Other/Multiracial & 654 & 60 & 10.0 [7.6, 12.4] \\
\midrule
\textbf{Poverty Status} & & & \\
\quad Low income ($<$1.3 PIR) & 2,130 & 224 & 10.2 [8.9, 11.5] \\
\quad Medium income (1.3--3.5 PIR) & 2,071 & 189 & 9.3 [8.0, 10.6] \\
\quad High income ($\geq$3.5 PIR) & 1,924 & 167 & 8.5 [7.2, 9.8] \\
\midrule
\textbf{BMI Category} & & & \\
\quad Underweight ($<$18.5) & 123 & 14 & 10.1 [5.1, 15.1] \\
\quad Normal (18.5--24.9) & 2,073 & 204 & 9.5 [8.2, 10.8] \\
\quad Overweight (25.0--29.9) & 1,618 & 141 & 8.5 [7.1, 9.9] \\
\quad Obese ($\geq$30.0) & 2,311 & 221 & 9.1 [7.9, 10.3] \\
\midrule
\textbf{Iron Supplement Use} & & & \\
\quad Non-user & 5,107 & 500 & 9.2 [8.4, 10.0] \\
\quad Supplement user & 1,018 & 80 & 7.9 [6.3, 9.5] \\
\bottomrule
\end{tabular}
\begin{flushleft}
\footnotesize{\textit{Notes:} IDWA = iron deficiency without anemia (ferritin $<$15~$\mu$g/L with hemoglobin $\geq$12~g/dL). PIR = poverty income ratio. CI = confidence interval. All estimates incorporate NHANES survey weights.}
\end{flushleft}
\end{table}

\newpage

% ============================================
% TABLE 3: REGRESSION RESULTS
% ============================================
\begin{table}[htbp]
\centering
\caption{Association Between Iron Supplement Use and Serum Ferritin Levels: Survey-Weighted Linear Regression Results}
\label{tab:regression}
\begin{tabular}{lccc}
\toprule
\textbf{Variable} & \textbf{Model 1} & \textbf{Model 2} & \textbf{Model 3} \\
& \textbf{(Unadjusted)} & \textbf{(Demographics)} & \textbf{(Fully Adjusted)} \\
\midrule
Iron supplement use & 0.081 & 0.049 & \textbf{0.062} \\
& [0.023, 0.140] & [$-$0.013, 0.110] & [0.001, 0.123] \\
& p=0.007$^{**}$ & p=0.120 & p=0.048$^{*}$ \\
\midrule
\multicolumn{4}{l}{\textit{Geometric mean ratio (\% difference)}} \\
\quad Iron supplement use & 8.4\% & 5.0\% & \textbf{6.4\%} \\
\midrule
\multicolumn{4}{l}{\textbf{Covariate Effects (Model 3 only)}} \\
Age, per year & --- & 0.006 & 0.004 \\
& & [$-$0.001, 0.013] & [$-$0.003, 0.011] \\
\midrule
\multicolumn{4}{l}{Race/Ethnicity (ref: Non-Hispanic White)} \\
\quad Mexican American & --- & $-$0.175 & $-$0.191 \\
& & [$-$0.274, $-$0.076] & [$-$0.292, $-$0.090] \\
\quad Other Hispanic & --- & $-$0.073 & $-$0.082 \\
& & [$-$0.194, 0.048] & [$-$0.205, 0.041] \\
\quad Non-Hispanic Black & --- & $-$0.148 & $-$0.188 \\
& & [$-$0.230, $-$0.066] & [$-$0.273, $-$0.103] \\
\quad Other/Multiracial & --- & $-$0.062 & $-$0.071 \\
& & [$-$0.168, 0.044] & [$-$0.179, 0.037] \\
\midrule
Poverty ratio, per unit & --- & 0.021 & 0.029 \\
& & [0.003, 0.039] & [0.011, 0.047] \\
\midrule
BMI, kg/m$^2$, per unit & --- & --- & 0.014 \\
& & & [0.008, 0.020] \\
\midrule
Model fit & & & \\
\quad N & 6,125 & 5,642 & 5,590 \\
\quad R$^2$ & 0.001 & 0.016 & 0.030 \\
\bottomrule
\end{tabular}
\begin{flushleft}
\footnotesize{\textit{Notes:} Outcome variable is natural log-transformed ferritin (ng/mL). Values are regression coefficients with 95\% confidence intervals. Model 1: Unadjusted. Model 2: Adjusted for age, race/ethnicity, and poverty income ratio. Model 3: Additionally adjusted for BMI. Geometric mean ratio calculated as $[\exp(\beta)-1] \times 100$. $^{*}$p$<$0.05; $^{**}$p$<$0.01; $^{***}$p$<$0.001.}
\end{flushleft}
\end{table}

\newpage

% ============================================
% TABLE 4: DOSE-RESPONSE ANALYSIS
% ============================================
\begin{table}[htbp]
\centering
\caption{Dose-Response Analysis: Association Between Iron Supplement Dose and Serum Ferritin}
\label{tab:dose_response}
\begin{tabular}{lcccccc}
\toprule
\textbf{Dose Category} & \textbf{n} & \textbf{\%} & \textbf{Coefficient} & \textbf{95\% CI} & \textbf{GMR} & \textbf{p-value} \\
\midrule
None (reference) & 5,107 & 81.5 & 0.000 & Reference & 1.00 & --- \\
Low ($<$18 mg/day) & 597 & 9.4 & $-$0.009 & [$-$0.090, 0.072] & 0.99 & 0.833 \\
Moderate (18--27 mg/day) & 332 & 5.5 & \textbf{0.207} & [0.103, 0.310] & \textbf{1.23} & $<$0.001$^{***}$ \\
High ($\geq$28 mg/day) & 89 & 3.5 & 0.023 & [$-$0.107, 0.153] & 1.02 & 0.727 \\
\midrule
\multicolumn{7}{l}{\textit{Overall test for trend: p$<$0.001}} \\
\midrule
\multicolumn{7}{l}{\textbf{Paired Comparisons (vs. None)}} \\
\quad Low vs. None & & & $-$0.009 & [$-$0.090, 0.072] & 0.99 & 0.833 \\
\quad Moderate vs. None & & & \textbf{0.207} & [0.103, 0.310] & 1.23 & $<$0.001$^{***}$ \\
\quad High vs. None & & & 0.023 & [$-$0.107, 0.153] & 1.02 & 0.727 \\
\midrule
\multicolumn{7}{l}{\textbf{Paired Comparisons (between doses)}} \\
\quad Moderate vs. Low & & & \textbf{0.216} & [0.091, 0.341] & 1.24 & 0.001$^{**}$ \\
\quad High vs. Low & & & 0.032 & [$-$0.118, 0.182] & 1.03 & 0.676 \\
\quad High vs. Moderate & & & $-$0.184 & [$-$0.348, $-$0.020] & 0.83 & 0.028$^{*}$ \\
\bottomrule
\end{tabular}
\begin{flushleft}
\footnotesize{\textit{Notes:} All models adjusted for age, race/ethnicity, poverty income ratio, and BMI. Outcome is natural log-transformed ferritin. GMR = Geometric Mean Ratio ($\exp(\beta)$). Dose categories: Low = over-the-counter multivitamin level; Moderate = prenatal vitamin level; High = therapeutic iron supplement level. $^{*}$p$<$0.05; $^{**}$p$<$0.01; $^{***}$p$<$0.001 (Bonferroni-corrected significance maintained for moderate dose effect).}
\end{flushleft}
\end{table}

\newpage

% ============================================
% TABLE 5: SENSITIVITY ANALYSES
% ============================================
\begin{table}[htbp]
\centering
\caption{Sensitivity Analyses: Robustness of Iron Supplement Association with Ferritin}
\label{tab:sensitivity}
\begin{tabular}{llccc}
\toprule
\textbf{Analysis} & \textbf{Description} & \textbf{Coefficient} & \textbf{95\% CI} & \textbf{p-value} \\
\midrule
\textbf{Primary Analysis} & Ferritin $<$15 ng/mL threshold & 0.062 & [0.001, 0.123] & 0.048$^{*}$ \\
\midrule
\textbf{Alternative IDWA Definitions} & & & & \\
\quad Conservative threshold & Ferritin $<$12 ng/mL & 0.058 & [$-$0.003, 0.119] & 0.062 \\
\quad Expanded threshold & Ferritin $<$20 ng/mL & 0.065 & [0.004, 0.126] & 0.037$^{*}$ \\
\quad Very expanded threshold & Ferritin $<$25 ng/mL & 0.071 & [0.010, 0.132] & 0.022$^{*}$ \\
\midrule
\textbf{Inflammation Adjustment} & & & & \\
\quad Excluded elevated CRP & CRP $\leq$10 mg/L & 0.058 & [$-$0.003, 0.119] & 0.062 \\
\quad Excluded any inflammation & CRP $\leq$5 mg/L & 0.055 & [$-$0.008, 0.118] & 0.088 \\
\midrule
\textbf{BMI Stratification} & & & & \\
\quad Normal weight & BMI 18.5--24.9 & 0.078 & [$-$0.012, 0.168] & 0.089 \\
\quad Overweight & BMI 25.0--29.9 & 0.054 & [$-$0.043, 0.151] & 0.276 \\
\quad Obese & BMI $\geq$30.0 & 0.061 & [$-$0.020, 0.142] & 0.141 \\
\quad \multicolumn{2}{l}{Interaction p-value} & \multicolumn{3}{c}{0.420} \\
\midrule
\textbf{Missing Data Approaches} & & & & \\
\quad Complete case analysis & n=5,590 & 0.062 & [0.001, 0.123] & 0.048$^{*}$ \\
\quad Multiple imputation & 5 imputations & 0.064 & [0.003, 0.125] & 0.041$^{*}$ \\
\midrule
\textbf{Cycle-Specific Analyses} & & & & \\
\quad Cycles D--F only & 2005--2010 & 0.058 & [$-$0.028, 0.144] & 0.188 \\
\quad Cycles I--J only & 2015--2018 & 0.068 & [$-$0.012, 0.148] & 0.096 \\
\quad Time trend test & Cycle$\times$supplement & \multicolumn{3}{c}{p=0.732} \\
\bottomrule
\end{tabular}
\begin{flushleft}
\footnotesize{\textit{Notes:} All models use fully adjusted specification (age, race/ethnicity, poverty ratio, BMI). Coefficients represent association between any iron supplement use and log-transformed ferritin. CRP = C-reactive protein. Interaction p-value tests whether supplement effect differs across BMI categories or survey cycles. $^{*}$p$<$0.05.}
\end{flushleft}
\end{table}

\end{document}
